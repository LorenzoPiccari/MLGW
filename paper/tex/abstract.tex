\begin{abstract}
%We apply Machine Learning methods to build a model designed to generate a gravitational waveform in the time domain as produced by a binary black hole coalescence. 
%Our model matches the accuracy of the state-of-the-art EOB models and has a tiny generation cost, equivalent to those of a fast Reduced Order Model.
%As the generation time does not depend on the lenght of the signal, the model is particularly suitable for long observation surveys, such as ET.
%Furthermore, it provides a closed form expression for the waveform and its gradient with respect to the orbital parameters. This could lead to a further improvement of the sampling algorithms employed for the parameter estimation.
%We train our model on a number of waveforms computed by \texttt{TEOBResumS} and we infer a relation between the waveform and the masses $m_1$, $m_2$ and (aligned) spins $s_1$, $s_2$ of the two BHs. 
%Our implementation is publicly available as a Python package \texttt{mlgw} as \url{https://pypi.org/project/mlgw/}.
%\texttt{mlgw} has all the features required for performing a full parameter estimation (including the waveform dependence on geometrical parameters). 
%We employ \texttt{mlgw} to analyse the public data from GWTC-1, the first GW transient catalog. Our results are largely compatible with those published by the LIGO-Virgo collaboration.

We apply machine learning methods to build a time-domain model for
gravitational waveforms from binary black hole mergers called \texttt{mlgw}.
The dimensionality of the problem is handled by representing the 
waveform's amplitude and phase using a principal component analysis.
We train \texttt{mlgw} on about $\mathcal{O}(10^3)$ \texttt{TEOBResumS} effective-one-body
waveforms with mass ratios $q\in[1,20]$ and aligned dimensionless
spins $\chi\in[-0.99,0.99]$. The resulting model is faithful to the
training set at the ${\sim}10^{-3}$ level (averaged on the parameter space), while speeding up the single waveform generation of a factor 10-50 (depending on the
binary mass and initial frequency.)
Furthermore, \texttt{mlgw} provides a closed form expression for the waveform and its gradient with respect to the orbital parameters; such an information might be for future improvements in GW data analysis.
\par
As demonstration of the capabilities of \texttt{mlgw} to perform a
full parameter estimation, we re-analyze the public data from the first
GW transient catalog (GWTC-1) to find consistent results at  a fraction of the cost.
Since the generation time does not depend on the length of the signal,
our model is particularly suitable for the analysis of long signals
and/or and in view of third-generation detectors.
Future applications include the analysis of waveform systematics and
model selection in parameter estimation.

\end{abstract}

